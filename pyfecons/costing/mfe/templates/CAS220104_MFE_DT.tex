% copied on Apr 3 from https://drive.google.com/file/d/1VJFKpj0qK2iz9v1q43rAanfx_hB-VWYN/view?usp=drive_link

\subsubsection*{Cost Category 22.01.04: Supplementary Heating Systems.}

Supplementary heating systems in fusion, such as neutral beam injection and radio frequency (RF) heating, are critical for achieving the necessary plasma conditions for fusion. Neutral beam injection involves injecting high-energy neutral atoms into the plasma, on the other hand, RF heating uses electromagnetic waves to transfer energy to the plasma. These systems are essential for heating the plasma to the extremely high temperatures required for fusion reactions to occur efficiently.\\

Supplementary heating systems contains costs associated with heating and current drive systems, including

\begin{itemize}
    \item 22.01.04.01 Neutral Beam Heating. Neutral Beam Injection (NBI): Injects high-energy neutral atoms (typically hydrogen or deuterium) into the plasma. These neutral atoms penetrate the magnetic fields and, upon ionization, transfer their energy to the plasma particles.
    \item 22.01.04.02 Ion Cyclotron Resonance Heating (ICRF). Uses radio frequency waves at the ion cyclotron frequency to transfer energy directly to the plasma ions.
    \item 22.01.04.03 Electron Cyclotron Resonance Heating (ECRH). Uses high-frequency microwaves (gyrotrons) at the electron cyclotron frequency to heat electrons. Also used for current drive (ECCD) and neoclassical tearing mode suppression.
    \item 22.01.04.04 Lower Hybrid Current Drive (LHCD). Uses lower hybrid waves (klystrons) to drive non-inductive plasma current. Primary steady-state current drive system for many reactor designs.
\end{itemize}


The concept presented here employs supplementary heating and current drive systems. By comparing with prior studies (see table \ref{tab:supp_heat}), a top-down estimation can be found for each. For a NBIPOWER MW NBI system, the cost is C22010401 M USD; for a ICRFPOWER MW ICRF system, C22010402 M USD; for a ECRHPOWER MW ECRH system, C22010403 M USD; and for a LHCDPOWER MW LHCD system, C22010404 M USD. The total is C220104__ MUSD.

\begin{table}[htbp]
    \centering
    \begin{tabular}{lcccrr}
        \hline
        System & Type & Power (MW) & \$/W (2009) & \$/W (2023) \\
        \hline
HEATING_TABLE_ROWS
        \hline
    \end{tabular}
    \caption{Power and Cost Data (Rounded to 3 Significant Figures)}
    \label{tab:supp_heat}
\end{table}

